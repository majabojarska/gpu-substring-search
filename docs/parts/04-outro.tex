\section{Wnioski i~podsumowanie}
% @everyone - Kuba
% @everyone
% Nie ma co się rozbijać na 100 sekcji

Po wykonaniu badań zaimplementowanych algorytmów można stwierdzić, że akceleracji przy obecnych implementacjach nie udało się uzyskać. Jest to skutkiem zarówno charakterystyki problemu, jak i pośrednika w postaci frameworka GPU.js, które posiada swoje ograniczenia.

Rozwiązaniem, które pomogłoby zwiększyć wydajność analizowanego algorytmu, czyli wariantu ,,brute force'', byłoby rozwiązanie hybrydowe. W~rozwiązaniu takim GPU byłoby odpowiedzialne za sprawdzenie dopasowania wzorca w~tekście dla każdej pozycji. CPU, za pomocą wielu wątków, agregowałby wyniki do oczekiwanej postaci tablicy wystąpień. Akceleracji uległby czas wykonania metody \texttt{filter}, która zajmuje zdecydowaną większość czasu jednego wykonania, co widać na rysunku~\ref{fig:profiler_gpu}. Akceleracja taka będzie efektywna dla dużych rozmiarów problemu, ponieważ w~analizie trzeba wziąć pod uwagę zróżnicowany czas komunikacji z~Workerami.

Do udanych można zaliczyć implementację algorytmu w 3 wariantach, gdyż zaproponowane rozwiązanie jest sprawne. Okazało się, że w odpowiednich warunkach akceleracja może być osiągnięta nawet w środowisku przeglądarki internetowej, które nie podlega kontroli ze strony specjalistycznego oprogramowania -- chociażby operującego na poziomie elementarnych instrukcji GPU środowiska CUDA. Niewątpliwie, wykorzystany framework GPU.js akcelerację umożliwia. Potencjalny stopień akceleracji jest zależny natomiast od rozważanego problemu.